\mytitlepage{Прикладной математики}{3}{Комьютерная графика}{Трассировка лучей}{ПМ-63}{Шепрут И.И.}{-}{Задорожный А.Г.}{2019}

\section{Цель работы}

Ознакомиться с основными аспектами метода трассировки лучей.

\section{Постановка задачи}

\noindent\normalsize{\begin{easylist}
\ListProperties(Hang1=true)
& Считывать из файла (в зависимости от варианта):
&& Тип объекта.
&& Координаты и размер объектов.
&& Параметры материала объектов.
& Выполнить трассировку первичных лучей.
& Добавить зеркальную плоскость и учесть отраженные лучи.
& Предусмотреть возможность включения/исключения объектов.
& Предусмотреть возможонсть изменения положения источника света.
\end{easylist}}

\section{Реализованные функции}

\noindent\normalsize{\begin{easylist}
\ListProperties(Hide=100, Hang1=true, 
	Progressive*=.5cm,%	
	Style1*=\textbullet ,%
	Style2*=\textopenbullet )
& \textbf{Два типа рендеринга.}
&& Ray Tracing --- обычная трассировка лучей, которая может отобразить картинку с малым количеством семплов. Трейсинг лучей останавливается при достижении материала с цветом, и рендерит вторичные лучи, только если попадает на преломляющую или зеркальную поверхность.
&& Path Tracing ---  методика рендеринга в компьютерной графике, которая стремится симулировать физическое поведения света настолько близко к реальному, насколько это возможно. Трейсинг лучей останавливается только при достижении предела отражений, или при достижении источника света.
& \textbf{Полиморфизм.}
&& Полиморфизм объектов --- имеется два типа объектов: object и shape.
&&& Object --- объект, котороый неразрывно связан со своим собственным материалом, например: небо, cubemap, текстурированный полигон.
&&& Shape --- объект, которому можно задать любой имеющийся материал, например, это объекты: сфера, цилиндр, полигон, треугольник, портал, контур (состоит из сфер и цилиндров).
&& Полиморфизм камер --- можно рендерить картинку совершенно различными камерами, которые наследуются от абстрактного класса базовой камеры и реализуют виртуальные методы. Таким образом уже реализованы камеры: перспективная, ортогональная, 360. Так же можно реализовать другие камеры, например: проекцию Панини, камера для создания cubemap.
&& Полиморфизм материалов --- любому объекту можно задать любой материал на его поверхности, например реализованы такие материалы, как: рассеивающий, отражающий, стекло, освещение. Аналогично для реализации материала надо наследоваться от базового класса материала и реализовать метод обработки луча материалом.
& \textbf{Точечные источники света} поддерживаются на уровне рендерера.
& \textbf{Рендеринг порталов.} Портал является shape, потому что можно задать какой материал будет на двух его задних частях. Он сам задается как полигон и две системы координат.
&& Поддерживается телепортирование освещения от точечных источников света через порталы. При этой телепортации учитывается, что некоторая часть света может не пройти через портал.
& \textbf{Считывание произвольной сцены из многоугольников, с анимацией, из \texttt{json} файла.}
& \textbf{Поддержка текстур.} Реализуется эта поддержка в классе текстурированного полигона.
& Рендеринг произвольных полигонов.
& Для path tracing камерой поддерживается симуляция прохождения луча через диафргаму, поэтому можно получить эффект depth of field: когда камера на чем-то сфокусирована, а остальная часть мира размыта.
& Рендеринг происходит в многопоточном режиме.
& Имеется возможность одновременно с обычным изображением рендерить изображение глубины.
& Имеется возможность сохранять отрендеренные изображения в png.
& При рендеринге пиксели выбираются случайным образом, чтобы максимально точно предсказывать итоговое время рендеринга, которое выводится на экран.
\end{easylist}}

\section{Пример реализованных функций}

\newcommand{\img}[2]{
\begin{center}
\includegraphics[width=#2\textwidth]{#1}	
\end{center}
}

\newcommand{\imgtwo}[4]{
\noindent\begin{center}
\includegraphics[width=#2\textwidth]{#1}
\includegraphics[width=#4\textwidth]{#3}
\end{center}
}

\subsection{Различие Ray tracing от Path tracing}

\imgtwo{img/standard_scene_ray_perspective.png}{0.49}{img/standard_scene_path_perspective.png}{0.49}

\imgtwo{img/standard_scene_2_ray_perspective.png}{0.49}{img/standard_scene_2_path_perspective.png}{0.49}

\subsection{Различные виды камер}

\subsubsection{Ортогональная}

\img{img/standard_scene_ray_orthogonal.png}{0.7}

\img{img/standard_scene_2_ray_orthogonal.png}{0.7}

\subsubsection{Камера на 360 градусов}

Эта камера создает такое изображение, которое запечатляет всю обстановку вокруг. Далее, по этому изображению можно получить сколько угодно изображений перспективной проекции, которые смотрят с разных углов при помощи различных программ для просмотра такого рода изображений.

Изображения в 360 градусов были просмотрены с помощью \href{https://www.chiefarchitect.com/products/360-panorama-viewer/}{www.chiefarchitect.com}.

\img{img/standard_scene_ray_360.png}{1}

\img{img/screen_360_1.png}{1}

\img{img/standard_scene_2_ray_360.png}{1}

\img{img/screen_360_2.png}{1}

\subsection{Телепортация освещения от точечного источника света через портал}

\img{img/portal_light_test1_1.png}{1}

\imgtwo{img/portal_light_test1_2.png}{0.49}{img/portal_light_test1_3.png}{0.49}

\subsection{Изображение глубины каждого пикселя}

\img{img/standard_scene_ray_perspective_depth.png}{0.7}

\subsection{Рендеринг произвольных полигонов}

\img{img/polygon_test.png}{0.7}

\subsection{Рендеринг сцен из предыдущей работы, заданных в json}

\imgtwo{img/oblique_portal.png}{0.49}{img/volumetric_portal_with_textures.png}{0.49}

\imgtwo{img/scale_portal.png}{0.49}{img/volumetric_portal_with_textures_depth.png}{0.49}

\subsection{Пример предсказания времени рендеринга}

\img{img/log.png}{0.7}

\section{Код программы}

\subsection{Файлы заголовков}

% \mycodeinput{c++}{code/fragment.h}{fragment.h}
% \mycodeinput{c++}{code/framebuffer.h}{framebuffer.h}
% \mycodeinput{c++}{code/opengl_common.h}{opengl\_common.h}
% \mycodeinput{c++}{code/plane.h}{plane.h}
% \mycodeinput{c++}{code/scene_reader.h}{scene\_reader.h}
% \mycodeinput{c++}{code/shader.h}{shader.h}

\subsection{Исходные файлы}

% \mycodeinput{c++}{code/main.cpp}{main.cpp}
% \mycodeinput{c++}{code/fragment.cpp}{fragment.cpp}
% \mycodeinput{c++}{code/framebuffer.cpp}{framebuffer.cpp}
% \mycodeinput{c++}{code/opengl_common.cpp}{opengl\_common.cpp}
% \mycodeinput{c++}{code/plane.cpp}{plane.cpp}
% \mycodeinput{c++}{code/scene_reader.cpp}{scene\_reader.cpp}
% \mycodeinput{c++}{code/shader.cpp}{shader.cpp}

\subsection{Шейдеры}

% \mycodeinput{c++}{code/draw.fragment.glsl}{draw.fragment.glsl}
% \mycodeinput{c++}{code/draw.vertex.glsl}{draw.vertex.glsl}
% \mycodeinput{c++}{code/drawpoly.fragment.glsl}{drawpoly.fragment.glsl}
% \mycodeinput{c++}{code/drawpoly.vertex.glsl}{drawpoly.vertex.glsl}
% \mycodeinput{c++}{code/merge.fragment.glsl}{merge.fragment.glsl}
% \mycodeinput{c++}{code/merge.vertex.glsl}{merge.vertex.glsl}