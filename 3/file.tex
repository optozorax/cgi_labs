\mytitlepage{Прикладной математики}{3}{Комьютерная графика}{Трассировка лучей}{ПМ-63}{Шепрут И.И.}{-}{Задорожный А.Г.}{2019}

\section{Цель работы}

\noindent Ознакомиться с основными аспектами метода трассировки лучей.

\section{Постановка задачи}

\noindent\normalsize{\begin{easylist}
\ListProperties(Hang1=true)
& Считывать из файла (в зависимости от варианта):
&& Тип объекта.
&& Координаты и размер объектов.
&& Параметры материала объектов.
& Выполнить трассировку первичных лучей.
& Добавить зеркальную плоскость и учесть отраженные лучи.
& Предусмотреть возможность включения/исключения объектов.
& Предусмотреть возможонсть изменения положения источника света.
\end{easylist}}

\section{Реализованные функции}

\noindent\normalsize{\begin{easylist}
\ListProperties(Hide=100, Hang=true, 
	Progressive*=.5cm,%	
	Style1*=\textbullet$\mspace{9mu}$,%
	Style2*=\textopenbullet$\mspace{9mu}$,%
	Style3*=$\diamond\mspace{9mu}$)
& \textbf{Полиморфизм.}
&& \textbf{Различные подходы к рендерингу.} --- одну и ту же сцену можно рендерить различными классами и получать соответственно различную скорость и фотореалистичность.
&&& \textbf{Ray Tracing} --- обычная трассировка лучей, которая может отобразить картинку с малым количеством семплов. Трейсинг лучей останавливается при достижении материала с цветом, и рендерит вторичные лучи, только если попадает на преломляющую или зеркальную поверхность.
&&& \textbf{Path Tracing} ---  методика рендеринга в компьютерной графике, которая стремится симулировать физическое поведения света настолько близко к реальному, насколько это возможно. Трейсинг лучей останавливается только при достижении предела отражений, или при достижении источника света.
&& \textbf{Полиморфизм объектов} --- имеется два типа объектов: object и shape.
&&& \textbf{Object} --- объект, котороый неразрывно связан со своим собственным материалом, например: небо, cubemap, текстурированный полигон.
&&& \textbf{Shape} --- объект, которому можно задать любой имеющийся материал, например, это объекты: сфера, цилиндр, полигон, треугольник, портал, контур (состоит из сфер и цилиндров).
&& \textbf{Полиморфизм камер} --- можно рендерить картинку совершенно различными камерами, которые наследуются от абстрактного класса базовой камеры и реализуют виртуальные методы. Таким образом уже реализованы камеры: перспективная, ортогональная, 360. Так же можно реализовать другие камеры, например: проекцию Панини, камера для создания cubemap.
&& \textbf{Полиморфизм материалов} --- любому объекту можно задать любой материал на его поверхности, например реализованы такие материалы, как: рассеивающий, отражающий, стекло, освещение. Аналогично для реализации материала надо наследоваться от базового класса материала и реализовать метод обработки луча материалом.
& \textbf{Точечные источники света} поддерживаются на уровне рендерера.
& \textbf{Рендеринг порталов.} Портал является shape, потому что можно задать какой материал будет на двух его задних частях. Он сам задается как полигон и две системы координат.
&& Поддерживается \textbf{телепортирование освещения} от точечных источников света через порталы. При этой телепортации учитывается, что некоторая часть света может не пройти через портал.
& \textbf{Считывание произвольной сцены из многоугольников, с анимацией, из \texttt{json} файла.}
& \textbf{Поддержка текстур.} Реализуется эта поддержка в классе текстурированного полигона.
& Рендеринг произвольных полигонов.
& Для path tracing камерой поддерживается симуляция прохождения луча через диафргаму, поэтому можно получить эффект \textbf{depth of field}: когда камера на чем-то сфокусирована, а остальная часть мира размыта.
& Рендеринг происходит в \textbf{многопоточном режиме}.
& Имеется возможность одновременно с обычным изображением рендерить изображение глубины.
& Имеется возможность сохранять отрендеренные изображения в png.
& При рендеринге пиксели выбираются случайным образом, чтобы максимально точно предсказывать итоговое время рендеринга, которое выводится на экран.
& Возможность задавать количество семплов для одного пикселя. В данной работе обычно выбирается 16 семплов для ray tracing, и 400 для path tracing.
\end{easylist}}

\section{Пример реализованных функций}

\newcommand{\img}[2]{
\begin{center}
\includegraphics[width=#2\textwidth]{#1}	
\end{center}
}

\newcommand{\imgtwo}[4]{
\noindent\begin{center}
\includegraphics[width=#2\textwidth]{#1}
\includegraphics[width=#4\textwidth]{#3}
\end{center}
}

\subsection{Различие Ray tracing от Path tracing}

\imgtwo{img/standard_scene_ray_perspective.png}{0.49}{img/standard_scene_path_perspective.png}{0.49}

\imgtwo{img/standard_scene_2_ray_perspective.png}{0.49}{img/standard_scene_2_path_perspective.png}{0.49}

Слева --- \textbf{Ray Tracing}, справа --- \textbf{Path Tracing}. На первой сцене присутствуют точечные источники света, а на второй точечный источник света присутствует только для \textbf{ray tracing}, поэтому картинка с \textbf{path tracing} выглядит такой темной. потому что весь свет там получится от светящегося полигона сверху, вероятность попадания в который невелика. 

Слева освещение получается только от фонового освещения и точечных источников света. Справа же освещение получается от трассировки вторичных лучей, которые идут в случайном направлении от рассеивающего материала. Благодаря этому получается очень фотореалистичная картинка, а так же такие эффекты, как: \textbf{глубина резкости} --- размытие ближних и дальних объектов, не попавших в фокус; \textbf{каустика} --- явление попадания большого количества света в одно место, её можно наблюдать под стекляной сферой на картинке с комнатой; освещение от зелёной и красной стены на полу рядом со стекляной сферой --- тоже эффект от вторичного трейсинга лучей; освещение потолка зелёным цветом --- от зелёной стены.

\subsection{Различные виды камер}

\subsubsection{Ортогональная}

\imgtwo{img/standard_scene_ray_orthogonal.png}{0.49}{img/standard_scene_2_ray_orthogonal.png}{0.49}

\subsubsection{Камера на 360 градусов}

Эта камера создает такое изображение, которое запечатляет всю обстановку вокруг. Далее, по этому изображению можно получить сколько угодно изображений перспективной проекции, которые смотрят с разных углов при помощи различных программ для просмотра такого рода изображений.

Изображения в 360 градусов были просмотрены с помощью \href{https://www.chiefarchitect.com/products/360-panorama-viewer/}{www.chiefarchitect.com}.

\img{img/standard_scene_ray_360.png}{1}

\img{img/screen_360_1.png}{1}

\img{img/standard_scene_2_ray_360.png}{1}

\img{img/screen_360_2.png}{1}

\subsection{Телепортация освещения от точечного источника света через портал}

Маленькой сферой показано положение точечного источника света. Темно-синий портал связан с красным, а голубой с оранжевым. Свет проходит через все порталы и освещает пол снизу. Так же некоторые части освещения от этого же источника света накладываются друг на друга, хотя источник света один. Эффект примерно такой же, как и от зеркал, но работает с порталами.

\img{img/portal_light_test1_1.png}{1}

\imgtwo{img/portal_light_test1_2.png}{0.49}{img/portal_light_test1_3.png}{0.49}

\subsection{Изображение глубины каждого пикселя}

\img{img/standard_scene_ray_perspective_depth.png}{0.7}

\subsection{Рендеринг произвольных полигонов}

\img{img/polygon_test.png}{0.7}

\subsection{Рендеринг сцен из предыдущей работы, заданных в json}

\imgtwo{img/oblique_portal.png}{0.49}{img/volumetric_portal_with_textures.png}{0.49}

\imgtwo{img/scale_portal.png}{0.49}{img/volumetric_portal_with_textures_depth.png}{0.49}

\subsection{Пример предсказания времени рендеринга}

Так как рендеринг изображения при помощи \textbf{path tracing} может занимать часы, иногда бывает необходимо знать точное время рендеринга. Специально для этого каждые $n$ отрендеренных пикселей собирается информация о времени рендеринга, находится процент завершенной работы и предсказывается время всего рендеринга. Так же для увеличения точности, все пиксели для рендеринга берутся в случайном порядке. Если же не делать этого случайного выбора пикселей, то у нас на изображении, у которого сверху небо, и соответственно, нет никаких вторичных лучей, будет очень быстро отрендерена верхняя часть, в то время, когда нижняя часть будет рендериться очень долго. Поэтому пиксели надо брать в случsайном порядке.

\img{img/log.png}{0.7}

\section{Код программы}

Код находится в открытом доступе по адресу: \href{https://github.com/optozorax/path-tracing}{path-tracing}. Зависимости: \href{https://github.com/optozorax/space_objects}{space-objects} --- для вычислений с системами координат; \href{https://github.com/nothings/stb}{stb} --- для сохранения результата в png; \href{https://github.com/nlohmann/json}{json}, \href{https://github.com/optozorax/portals_opengl}{portals-opengl} --- для работы кода по считыванию сцены из json.

\subsection{Файлы заголовков}

\mycodeinput{c++}{code/include/pt/basics.h}{basics.h}
\mycodeinput{c++}{code/include/pt/camera.h}{camera.h}
\mycodeinput{c++}{code/include/pt/image.h}{image.h}
\mycodeinput{c++}{code/include/pt/object.h}{object.h}
\mycodeinput{c++}{code/include/pt/poly.h}{poly.h}
\mycodeinput{c++}{code/include/pt/pt.h}{pt.h}
\mycodeinput{c++}{code/include/pt/renderer.h}{renderer.h}
\mycodeinput{c++}{code/include/pt/pt2easybmp.h}{pt2easybmp.h}

\subsubsection{camera/}

\mycodeinput{c++}{code/include/pt/camera/360.h}{360.h}
\mycodeinput{c++}{code/include/pt/camera/orthogonal.h}{orthogonal.h}

\subsubsection{material/}

\mycodeinput{c++}{code/include/pt/material/light.h}{light.h}
\mycodeinput{c++}{code/include/pt/material/reflect.h}{reflect.h}
\mycodeinput{c++}{code/include/pt/material/refract.h}{refract.h}
\mycodeinput{c++}{code/include/pt/material/scatter.h}{scatter.h}

\subsubsection{object/}

\mycodeinput{c++}{code/include/pt/object/cubemap.h}{cubemap.h}
\mycodeinput{c++}{code/include/pt/object/scene.h}{scene.h}
\mycodeinput{c++}{code/include/pt/object/sky.h}{sky.h}
\mycodeinput{c++}{code/include/pt/object/texture_polygon.h}{texture\_polygon.h}

\subsubsection{shape/}

\mycodeinput{c++}{code/include/pt/shape/contour.h}{contour.h}
\mycodeinput{c++}{code/include/pt/shape/cylinder.h}{cylinder.h}
\mycodeinput{c++}{code/include/pt/shape/polygon.h}{polygon.h}
\mycodeinput{c++}{code/include/pt/shape/portals.h}{portals.h}
\mycodeinput{c++}{code/include/pt/shape/sphere.h}{sphere.h}
\mycodeinput{c++}{code/include/pt/shape/triangle.h}{triangle.h}

\subsection{Исходные файлы}

\mycodeinput{c++}{code/src/360.cpp}{360.cpp}
\mycodeinput{c++}{code/src/basics.cpp}{basics.cpp}
\mycodeinput{c++}{code/src/camera.cpp}{camera.cpp}
\mycodeinput{c++}{code/src/contour.cpp}{contour.cpp}
\mycodeinput{c++}{code/src/cubemap.cpp}{cubemap.cpp}
\mycodeinput{c++}{code/src/cylinder.cpp}{cylinder.cpp}
\mycodeinput{c++}{code/src/image.cpp}{image.cpp}
\mycodeinput{c++}{code/src/light.cpp}{light.cpp}
\mycodeinput{c++}{code/src/orthogonal.cpp}{orthogonal.cpp}
\mycodeinput{c++}{code/src/poly.cpp}{poly.cpp}
\mycodeinput{c++}{code/src/polygon.cpp}{polygon.cpp}
\mycodeinput{c++}{code/src/portals.cpp}{portals.cpp}
\mycodeinput{c++}{code/src/pt2easybmp.cpp}{pt2easybmp.cpp}
\mycodeinput{c++}{code/src/reflect.cpp}{reflect.cpp}
\mycodeinput{c++}{code/src/refract.cpp}{refract.cpp}
\mycodeinput{c++}{code/src/renderer.cpp}{renderer.cpp}
\mycodeinput{c++}{code/src/scatter.cpp}{scatter.cpp}
\mycodeinput{c++}{code/src/scene.cpp}{scene.cpp}
\mycodeinput{c++}{code/src/sky.cpp}{sky.cpp}
\mycodeinput{c++}{code/src/sphere.cpp}{sphere.cpp}
\mycodeinput{c++}{code/src/texture_polygon.cpp}{texture\_polygon.cpp}
\mycodeinput{c++}{code/src/triangle.cpp}{triangle.cpp}

\subsection{Сцены}

\mycodeinput{c++}{code/examples/hello_world.cpp}{hello\_world.cpp}
\mycodeinput{c++}{code/examples/polygon_test.cpp}{polygon\_test.cpp}
\mycodeinput{c++}{code/examples/portal_light_test.cpp}{portal\_light\_test.cpp}
\mycodeinput{c++}{code/examples/scene_raytracing.cpp}{scene\_raytracing.cpp}
\mycodeinput{c++}{code/examples/standard_scene_2.cpp}{standard\_scene\_2.cpp}
\mycodeinput{c++}{code/examples/standart_scene.cpp}{standard\_scene.cpp}